\documentclass{aip-cp}
\usepackage{amssymb}
\usepackage{amsmath}
\usepackage{mathtools}
\usepackage{listings}
\usepackage[numbers]{natbib}
\usepackage{graphicx}
\begin{document}
\title{Consistent Pure Language with Effect Type System for Certified Applications and Higher Languages}

\author[aff1,aff2]{Maxim Sokhatsky\corref{cor1}}
\eaddress[url]{https://groupoid.space}
\author[aff1]{Pavlo Maslianko\corref{cor3}}

\affil[aff1]{National Technical University of Ukraine "Igor Sikorsky Kyiv Polytechnical Institute"}
\affil[aff2]{Groupoid Infinity}
\corresp[cor1]{maxim@synrc.com}
\corresp[cor3]{mppdom@i.ua}

\maketitle
\tableofcontents
\vspace{1cm}
\begin{abstract}
This paper presents design of Om language and implementation of its type checker
and bytecode extraction to Erlang. Om is an intermediate language
based on a pure type system with infinite
number of universes, so it is known to be consistent in dependent type theory.
The typechecker can be switched between predicative and impredicative hierarchies
of universes. The need to natively support Erlang platform dictated the look and feel of this work.
This system is expected to be usable as trusted core for certified
applications which could be run inside Erlang virtual machines LING and BEAM.
The syntax is compatible with Morte language and supports its base
library, however it extends the indexed universes.
We show how to program in this environment and link with Erlang
inductive and coinductive free structures. A very basic prelude library is shipped
as a part of the work. We briefly describe the top-level
language which compiles to pure type system core. In the Results section we will
show that lambda evaluation performance on BEAM virtual machine is acceptable to run
production software.
\end{abstract}

\section{Introduction}
As a part of verification and validation process, according to IEEE\footnote{IEEE Std 1012-2016  --- V\&V Software verification and validation} standard
and ESA\footnote{ESA PSS-05-10 1-1 1995 -- Guide to software verification and validation} regulatory document,
exists a lot of tools and approaches. Most advanced techniques
involves mathematical languages and notations. The age of verified math was
started by de Bruin AUTOMATH prover and today we have Coq, HOL/Isabell, F* languages
based on so known Calculus of Inductive Constructions (CiC)\footnote{Christine Paulin-Mohring. Introduction to the Calculus of Inductive Constructions.}\cite{Mohring15}.
The core of CiC is Calculus of Constructions\footnote{T. Coquand, Ǵerard Huet. The calculus of constructions} (CoC)\cite{Coq88}.
The further developing led to Lambda Cube\footnote{H. Barendregt. Lambda Cacluli with Types}\cite{Henk93} and Pure Type Systems (Henk\footnote{Simon Peyton Jones, Erik Meijer. Henk: a typed intermediate language.},Morte\footnote{Gabriel Gonzalez. Haskell Morte Library}).
Pure Type Systems are custom languages based on CoC with single Pi-type and possible other extensions.
The known extensions are ECC\footnote{Zhaohui Luo. An Extended Calculus of Constructions}, K-rules\footnote{Gilles Barthe. Extensions of pure type systems.}.
The main motivation of Pure Type Systems is an easy reasoning,
proving, strong normalization and trusted external verification.
Imagine one can implement their own typechecker to run certified
programs retrieved over untrusted channels. The applications of
such minimal cores are: 1) Blockchain smart-contract languages,
2) certified applications kernels, 3) payment processing, etc.

\subsection{Generating Trusted Programs}
As both specifications and implemntations are done in typed
language with full lambda cube features (dependent types) we
can extract target implementation of certified program just
in any programming language. These languages could be so
primitive as untyped lambda calculus which manifests (usually implements) as
untyped interpreters (JavaScript, Erlang, PyPy, LuaJIT, K).
The most advanced usage is code generation to higher-level
languages such as C++ and Rust (whish is already could be takes
as language with trusted features on memory and variable
accessing, linear types, etc.). In this work we presents a
simple code extraction to Erlang programming language as target interpreter.
However we have working also on C++ and Rust targets as well.

\subsection{System Architecture}
As a core type system, the {\bf PTS$^{\infty}$} is intendent to have multiple fron-end languages.
Such languages are based on CiC with extensions, like [0,1] Homotpy Core, Stream Calculus (Futhark)
or Pi caclulus (elements is different languages such as Erlang, Ling, Rust).
Different languages desugar to {\bf PTS$^{\infty}$} as an intermediate language before
extracting to target language\footnote{Note that extracting from [0,1] Homotopy Core is an open problem}.

\begin{figure}[h]
  \centerline{\includegraphics[scale=0.28]{static2}}
  \caption{Process of Model Verifications}
\end{figure}

\subsection{Place among other Languages}
\begin{table}[h]
\caption{List of languages, tried as verification targets}
\label{tab:a}
\tabcolsep7pt\begin{tabular}{lcccc}
\hline
\tch{1}{c}{b}{Target Language} & \tch{1}{c}{b}{Class} & \tch{1}{c}{b}{Higher\\ Languages} & \tch{1}{c}{b}{Type\\ Theory}\\
\hline
C++        & compiler/native      & HNC & System F\\
Rust       & compiler/native      & HNC & System F\\
JVM        & interpreter/native   & Java    & F-sub\footnote{System F wit bounded quantification}\\
JVM        & interpreter/native   & Scala   & System F-omega\\
GHC Core   & compiler/native      & Haskell & System D\\
GHC Core   & compiler/native      & Morte   & CoC\\
Haskell    & compiler/native      & Coq     & CiC\\
OCaml      & compiler/native      & Coq     & CiC\\
{\bf BEAM} & {\bf interpreter} & {\bf Om}   & {\bf PTS$^\infty$} \\
O          & interpreter          & Om  & PTS^\infty \\
K          & interpreter          & Q   & Applicative \\
PyPy       & interpreter/native   & N/A & ULC \\
LuaJIT     & interpreter/native   & N/A & ULC \\
JavaScript & interpreter/native & PureScript & System F\\
\hline
\end{tabular}
\end{table}

\section{Pure Type System as Intermediate Language}

\paragraph{}
The Om language is a dependently typed lambda calculus, an extension of Barendregt'
and Coquand Calculus of Constructions with predicative hierarchy of indexed universes.
There is no fixpoint axiom, so there is no infinite term dependence.
This means theory is fully consistent and recursion modeled in this
system could only be stricted.

\paragraph{}
All terms respect ranking $Axioms$ inside sequence of universes $Sorts$ and complexity of the
dependent term is equal to maximum complexity of term and its dependency $Rules$. The universe
system is completely described by the following PTS notation (due to Barendregt):

$$
\begin{cases}
Sorts = Type.\{i\},\ i : Nat\\
Axioms = Type.\{i\} : Type.\{inc\ i\}\\
Rules = Type.\{i\} \leadsto Type.\{j\} : Type.\{max\ i\ j\}\\
\end{cases}
$$

\paragraph{}
The Om language is based on Henk languages described first
by Erik Meijer and Simon Peyton Jones in 1997\cite{Erik97}. Leter on in 2015 Morte impementation
of Henk design appeared in Haskell, using Boem-Berrarducci encoding of non-recursive lambda terms.
It is based only on one type constructor $\Pi$, its intro $\lambda$ and $apply$ eliminator,
infinity number of universes, and $\beta$-reduction. The design of Om language resemble
Henk and Morte both design and implementation. This language indended to be small,
concise, easy provable and able to produce verifiable peace of code that can be
distributed over the networks, compiled at target with safe trusted linkage.

\subsection{BNF}

Om syntax is compatible with $\lambda C$ Coquand's Calculus of Constructions presented
in Morte and Henk languages. However it has extension in a part of specifying
universe index as a {\bf Nat} number. Traditionally we present the language in Backus-Naur form.

\vspace{0.5cm}
\begin{lstlisting}[mathescape=true]
   <> ::= #option
    I ::= #identifier
    U ::= * < #number >
    O ::= U
        | I | ( O ) | O O | O $\rightarrow$ O
        | $\lambda$ ( I : O ) $\rightarrow$ O
        | $\forall$ ( I : O ) $\rightarrow$ O
\end{lstlisting}

Equivalent tree encoding (in cubical syntax) for parsed terms is following:

\vspace{0.5cm}
\begin{lstlisting}[mathescape=true]
data pts = star (n: nat)
         | var (n: name)
         | app (f a: pts)
         | lambda (x: name) (d c: pts)
         | pi (x: name) (d c: pts)
\end{lstlisting}

\subsection{Universes}

As Om has infinite number of universes it should include metatheoretical Nat
inductive type in its core. Om supports predicative and impredicative hierarchies.

$$
U_0 : U_1 : U_2 : U_3 : ...
$$

$U_0$ --- propositions\\
$U_1$ --- sets\\
$U_2$ --- types\\
$U_3$ --- kinds\\

\begin{equation}
\tag{I}
\dfrac
{}
{Nat}
\end{equation}

\begin{equation}
\tag{S}
\dfrac
{o : Nat}
{Type_o}
\end{equation}

You may check if a term is an universe with the star function.
If argument is not an universe it returns $\{error,\_\}$.

\begin{lstlisting}[mathescape=true]
star (:star,N) $\rightarrow$ N
star _ $\rightarrow$ (:error, "*")
\end{lstlisting}

\subsection{Predicative Universes}

All terms obey the A ranking inside the sequence of S universes,
and the complexity R of the dependent term is equal to a maximum of
the term's complexity and its dependency.
Note that predicative universes are incompatible with Church lambda term encoding.
You choose either predicative or impredicative universes by a typechecker parameter.


\[
\tag{$A_1$}
\dfrac{i: Nat, j: Nat, i < j}{Type_i : Type_j}
\]

\[
\tag{$R_1$}
\dfrac{i : Nat, j : Nat}{Type_i \rightarrow Type_j : Type_{max(i,j)} }
\]

\subsection{Impredicative Universes}
Propositional contractible bottom space is the only
available extension to predicative hierarchy which doesn't lead to inconsistency.
However there is another option to have infinite
impredicative hierarchy.

\begin{equation}
\tag{$A_2$}
\dfrac
{i: Nat}
{Type_i : Type_{i+1}}
\end{equation}

\begin{equation}
\tag{$R_2$}
\dfrac
{i : Nat,\ \ \ \ j : Nat}
{Type_i \rightarrow Type_{j} : Type_{j}}
\end{equation}

\subsection{Hierarchy Switching}
H returns the target Universe of B term dependence on A. There are two dependence rules known
as the predicative one and the impredicative one which return max universe or universe of last term respectively.

\begin{lstlisting}[mathescape=true]
dep A B impredicative $\rightarrow$ B
dep A B predicative   $\rightarrow$ max A B

h Arg Out $\rightarrow$ dep Arg Out om:hierarchy(impredicative)
\end{lstlisting}

\subsection{Contexts}

The contexts model a dictionary with variables for typechecker.
It can be typed as simple $List\ (Prod\ String\ Term)$. The elimination
rule is not given here as in our implementation the whole dictionary
is destroyed after typechecking.

\begin{equation}
\tag{Ctx-formation}
\dfrac
{}
{\Gamma : Context}
\end{equation}

\begin{equation}
\tag{Ctx-intro_1}
\dfrac
{\Gamma : Context}
{Empty : \Gamma}
\end{equation}

\begin{equation}
\tag{Ctx-intro_2}
\dfrac
{A : Type_i,\ \ \ \ x : A,\ \ \ \ \Gamma : Context}
{(x : A)\ \vdash\ \Gamma : Context}
\end{equation}

\subsection{Single Axiom Language}

This langauge is called one axiom language (or pure) as eliminator
and introduction adjoint functors inferred from type formation rule.
The only computation rule of Pi type is called beta-reduction.

\begin{equation}
\tag{$\Pi$-formation}
\dfrac
{x:A \vdash B : Type}
{\Pi\ (x:A) \rightarrow B : Type}
\end{equation}

\begin{equation}
\tag{$\lambda$-intro}
\dfrac
{x:A \vdash b : B}
{\lambda\ (x:A) \rightarrow b : \Pi\ (x: A) \rightarrow B }
\end{equation}

\begin{equation}
\tag{$App$-elimination}
\dfrac
{f: (\Pi\ (x:A) \rightarrow B)\ \ \ a: A}
{f\ a : B\ [a/x]}
\end{equation}

\begin{equation}
\tag{$\beta$-computation}
\dfrac
{x:A \vdash b: B\ \ \ a:A}
{(\lambda\ (x:A) \rightarrow b)\ a = b\ [a/x] : B\ [a/x]}
\end{equation}

\vspace{0.5cm}
This language could be embedded in itself and used as Logical Framework for the Pi type:

\begin{lstlisting}[mathescape=true]
record Pi (A: Type) :=
       (intro:  (A $\rightarrow$ Type) $\rightarrow$ Type)
       (lambda: (B: A $\rightarrow$ Type) $\rightarrow$ pi A B $\rightarrow$ intro B)
       (app:    (B: A $\rightarrow$ Type) $\rightarrow$ intro B $\rightarrow$ pi A B)
       (applam: (B: A $\rightarrow$ Type) (f: pi A B) $\rightarrow$ (a: A) $\rightarrow$
                Path (B a) ((app B (lambda B f)) a) (f a))
       (lamapp: (B: A $\rightarrow$ Type) (p: intro B) $\rightarrow$
                Path (intro B) (lambda B ($\lambda$ (a:A) $\rightarrow$ app B p a)) p)
\end{lstlisting}

The theorems intentionally left blank, as it proofs could be taken from various sources.

We extend the $PTS\infty$ with remote AST node which means remote file loading
from trusted storage, anyway this will be checked by typechecker. We deny recursion
over remote node. We also add index to var for simplified de Bruijn indexes,
we allow overlapped names with tags, incremented on each new occurance.

\begin{lstlisting}[mathescape=true]
data om = star             (n: nat)
        | var    (n: name) (n: nat)
        | remote (n: name) (n: nat)
        | app                       (f a: om)
        | lambda (x: name)          (d c: om)
        | arrow                     (d c: om)
        | pi     (x: name)          (d c: om)
\end{lstlisting}

\subsection{Functions}
Func returns true if the argument is a functional space. Otherwise it returns $\{error,\_\}$.

\begin{lstlisting}[mathescape=true]
func ((:forall,),(I,O)) $\rightarrow$ true
func T                  $\rightarrow$ (:error,(:forall,T))
\end{lstlisting}

\subsection{Variables}
Var returns true if the var $N$ is defined in dictionary $B$. Otherwise it returns $\{error,\_\}$.

\begin{lstlisting}[mathescape=true]
var N B                $\rightarrow$ var N B (proplists:is_defined N B)
var N B true           $\rightarrow$ true
var N B false          $\rightarrow$ (:error,("free var",N,proplists:get_keys(B)))
\end{lstlisting}

\subsection{Shift}
Shift renames var N in B. Renaming means adding an 1 to an index of that name.

\begin{lstlisting}[mathescape=true]
sh (:var,(N,I)),N,P) when I>=P  $\rightarrow$ (var,(N,I+1))
sh ((:forall,(N,0)),(I,O)),N,P) $\rightarrow$ ((:forall,(N,0)),sh I N P,sh O N P+1)
sh ((:lambda,(N,0)),(I,O)),N,P) $\rightarrow$ ((:lambda,(N,0)),sh I N P,sh O N P+1)
sh (Q,(L,R),N,P)                $\rightarrow$ (Q,sh L N P,sh R N P)
sh (T,N,P)                      $\rightarrow$ T
\end{lstlisting}

\subsection{Substitution}

\begin{equation}
\tag{subst}
\dfrac
{\pi_1 : A\ \ \ \ u:A \vdash \pi_2 : B}
{[\pi_1/u]\ \pi_2 : B}
\end{equation}

\begin{lstlisting}[mathescape=true]
sub Term Name Value               $\rightarrow$ sub Term Name Value 0
sub (:arrow,         (I,O)) N V L $\rightarrow$ (:arrow,         sub I N V L,sub O N V L);
sub ((:forall,(N,0)),(I,O)) N V L $\rightarrow$ ((:forall,(N,0)),sub I N V L,sub O N(sh V N 0)L+1)
sub ((:forall,(F,X)),(I,O)) N V L $\rightarrow$ ((:forall,(F,X)),sub I N V L,sub O N(sh V F 0)L)
sub ((:lambda,(N,0)),(I,O)) N V L $\rightarrow$ ((:lambda,(N,0)),sub I N V L,sub O N(sh V N 0)L+1)
sub ((:lambda,(F,X)),(I,O)) N V L $\rightarrow$ ((:lambda,(F,X)),sub I N V L,sub O N(sh V F 0)L)
sub (:app,           (F,A)) N V L $\rightarrow$ (:app,sub F N V L,sub A N V L)
sub (:var,           (N,L)) N V L $\rightarrow$ V
sub (:var,           (N,I)) N V L when I>L $\rightarrow$ (:var,(N,I-1))
sub T                       _ _ _ $\rightarrow$ T.
\end{lstlisting}

\subsection{Type Checker}

For sure in a pure system we should be careful with {\bf :remote} AST node. Remote
AST nodes like {\bf \#List/Cons or \#List/map} are remote links to files. So using
trick one should desire circular dependency over {\bf :remote}. This is denied in
the system. The same notes apply to normalization and definitional equality.

\begin{lstlisting}[mathescape=true]
type (:star,N)               D $\rightarrow$ (:star,N+1)
type (:var,(N,I))            D $\rightarrow$ let true = var N D in keyget N D I
type (:remote,N)             D $\rightarrow$ cache type N D
type (:arrow,(I,O))          D $\rightarrow$ (:star,h(star(type I D)),star(type O D))
type ((:forall,(N,0)),(I,O)) D $\rightarrow$ (:star,h(star(type I D)),star(type O [(N,norm I)|D]))
type ((:lambda,(N,0)),(I,O)) D $\rightarrow$ let star (type I D)
                                      NI = norm I
                                   in ((:forall,(N,0)),(NI,type(O,[(N,NI)|D])))
type (:app,(F,A))            D $\rightarrow$ let T = type(F,D)
                                      true = func T
                                      ((:forall,(N,0)),(I,O)) = T
                                      Q = type A D
                                      true = eq I Q
                                   in norm (subst O N A)
\end{lstlisting}

\subsection{Normalization}

Normalization perform substitutions on applications to functions searching
over all function spaces, performing recursive normalization over the lambda and pi nodes.

\begin{lstlisting}[mathescape=true]
norm :none                   $\rightarrow$ :none
norm :any                    $\rightarrow$ :any
norm (:app,(F,A))            $\rightarrow$ case norm F of
                                ((:lambda,(N,0)),(I,O)) $\rightarrow$ norm (subst O N A)
                                                     NF $\rightarrow$ (:app,(NF,norm A)) end
norm (:remote,N)             $\rightarrow$ cache (norm N [])
norm (:arrow,         (I,O)) $\rightarrow$ ((:forall,("_",0)),(norm I,norm O))
norm ((:forall,(N,0)),(I,O)) $\rightarrow$ ((:forall,(N,0)),    (norm I,norm O))
norm ((:lambda,(N,0)),(I,O)) $\rightarrow$ ((:lambda,(N,0)),    (norm I,norm O))
norm T                       $\rightarrow$ T
\end{lstlisting}

\subsection{Definitional Equality}

Definitional Equality simply checks the equality of Erlang terms.

\begin{lstlisting}[mathescape=true]
eq ((:forall,("_",0)), X) (:arrow,Y)     $\rightarrow$ eq X Y
eq (:app,(F1,A1))         (:app,(F2,A2)) $\rightarrow$ let true = eq F1 F2 in eq A1 A2
eq (:star,N)              (:star,N)      $\rightarrow$ true
eq (:var,(N,I))           (:var,(N,I))   $\rightarrow$ true
eq (:remote,N)            (:remote,N)    $\rightarrow$ true
eq ((:farall,(N1,0)),(I1,O1))
   ((:forall,(N2,0)),(I2,O2)) $\rightarrow$
   let true = eq I1 I2
    in eq O1 (subst (shift O2 N1 0) N2 (:var,(N1,0)) 0)
eq ((:lambda,(N1,0)),(I1,O1))
   ((:lambda,(N2,0)),(I2,O2)) $\rightarrow$
   let true = eq I1 I2
    in eq O1 (subst (shift O2 N1 0) N2 (:var,(N1,0)) 0)
eq (A,B)                      $\rightarrow$ (:error,(:eq,A,B))
\end{lstlisting}

\section{Language Usage}

In this section we will lift PTS system to MLTT system by defining
$Sigma$ and $Equ$ types using only $Pi$ type. We will use Boehm inductive
dependent encoding.

\subsection{Sigma Type}

\paragraph{}
Here we will show you some examples of Om Language usage.
The PTS system is extremely powerful even without Sigma type. But
we can encode {\bf Sigma} type similar how we encode {\bf Prod}
tuple pair in Boehm encoding. Let's formulate Sigma type as an inductive type:

\begin{lstlisting}[mathescape=true]
data Sigma (A: Type) (P: A -> Type) (x: A): Type =
     (intro: P x -> Sigma A P)
\end{lstlisting}

\begin{lstlisting}[mathescape=true]
-- Sigma/@
   \ (A: *)
-> \ (P: A -> *)
-> \ (n: A)
-> \/ (Exists: *)
-> \/ (Intro: A -> P n -> Exists)
-> Exists

-- Sigma/Intro
   \ (A: *)
-> \ (P: A -> *)
-> \ (x: A)
-> \ (y: P x)
-> \ (Exists: *)
-> \ (Intro: \/ (x:A) -> P x -> Exists)
-> Intro x y

-- Sigma/fst
   \ (A: *)
-> \ (B: A -> *)
-> \ (n: A)
-> \ (S: #Sigma/@ A B n)
-> S A ( \(x: A) -> \(y: B n) -> x)

-- Sigma/snd
   \ (A: *)
-> \ (B: A -> *)
-> \ (n: A)
-> \ (S: #Sigma/@ A B n)
-> S (B n) ( \(_: A) -> \(y: B n) -> y )
\end{lstlisting}

\begin{lstlisting}[mathescape=true]
> om:fst(om:erase(om:norm(om:a("#Sigma/test.fst")))).
{{$\lambda$,{'Succ',0}},
 {any,{{$\lambda$,{'Zero',0}},{any,{var,{'Zero',0}}}}}}
\end{lstlisting}

For using {\bf Sigma} type for Logic purposes one should change the
home Universe of the type to {\bf Prop}. Here it is:

\begin{lstlisting}[mathescape=true]
data Sigma (A: Prop) (P: A -> Prop): Prop =
     (intro: (x:A) (y:P x) -> Sigma A P)
\end{lstlisting}

\subsection{Equality Type}

Another example of expressiveness is Equality type a la Martin-Löf.

\begin{lstlisting}[mathescape=true]
data Equ (A: Type): A -> A -> Type :=
     (refl (a: A): Equ A a a)
\end{lstlisting}

\begin{lstlisting}[mathescape=true]
-- Equ/@
   \ (A: *)
-> \ (x: A)
-> \ (y: A)
-> \/ (Equ: A -> A -> *)
-> \/ (Refl: \/ (z: A) -> Equ z z)
-> Equ x y

-- Equ/Refl
   \ (A: *)
-> \ (x: A)
-> \ (Equ: A -> A -> *)
-> \ (Refl: \/ (z: A) -> Equ z z)
-> Refl x
\end{lstlisting}

You cannot construct a lambda that will check different values of A type for equality,
however you may want to use built-in definitional equality and
normalization feature of typechecker to actually compare two values:

\begin{lstlisting}[mathescape=true]
> om:print(
  om:type(
  om:a("(\\ (z: #Equ/@ #Nat/@ #Nat/One #Nat/One) -> #Prop/True)"++
       " (#Equ/Refl #Nat/@ (#Nat/Succ #Nat/Zero))"))).
   \/ (True: *0)
-> \/ (Intro: True)
-> True
ok

> om:print(
  om:type(
  om:a("(\\ (z: #Equ/@ #Nat/@ #Nat/One #Nat/One) -> #Prop/True)"++
       " (#Equ/Refl #Nat/@ #Nat/Zero)"))).
** exception error: no match of right hand side value
   {error,{"==",
          {app,{{var,{'Succ',0}},{var,{'Zero',0}}}},
          {var,{'Zero',0}}}}
\end{lstlisting}

\subsection{Effect Type System}

\paragraph{}
This work is expected to compile to a limited number of target platforms. For now Erlang, Haskell and LLVM are awaiting.
Erlang version is expected to be useful both on LING and BEAM Erlang virtual machines. This language
allows you to define trusted operations in System F and extract this routines to Erlang/OTP platform
and plug as trusted resourses. As example we also provide infinite coinductive process creation
and inductive shell that linked to Erlang/OTP IO functions directly.

\paragraph{IO protocol}
We can construct in pure type system the state machine based on (co)free
monads driven by IO protocols. Assume that String is a List of Nat (as it is in Erlang natively),
and three external constructors: getLine, putLine and pure. We need to
put correspondent implementations on host platform as parameters to perform the actual IO.

\begin{lstlisting}
String: Type = List Nat
data IO: Type =
     (getLine: (String -> IO) -> IO)
     (putLine: String -> IO)
     (pure: () -> IO)
\end{lstlisting}

\subsubsection{Infinity I/O Type}

Infinity I/O Type Spec.

\begin{lstlisting}[mathescape=true]
-- IOI/@
   \ (r : *)
-> \/ (x : *)
-> (\/ (s : *) -> s -> (s -> #IOI/F r s) -> x)
-> x

-- IOI/F
   \ (a : *)
-> \ (State : *)
-> \/ (IOF : *)
-> \/ (PutLine_ : #IOI/data -> State -> IOF)
-> \/ (GetLine_ : (#IOI/data -> State) -> IOF)
-> \/ (Pure_ : a -> IOF)
-> IOF

-- IOI/MkIO
   \ (r : *)
-> \ (s : *)
-> \ (seed : s)
-> \ (step : s -> #IOI/F r s)
-> \ (x : *)
-> \ (k : forall (s : *) -> s -> (s -> #IOI/F r s) -> x)
-> k s seed step

-- IOI/data
#List/@ #Nat/@
\end{lstlisting}

Infinite I/O Sample Program.

\begin{lstlisting}[mathescape=true]
-- Morte/corecursive
( \ (r: *1)
 -> ( (((#IOI/MkIO r) (#Maybe/@ #IOI/data)) (#Maybe/Nothing #IOI/data))
    ( \ (m: (#Maybe/@ #IOI/data))
     -> (((((#Maybe/maybe #IOI/data) m) ((#IOI/F r) (#Maybe/@ #IOI/data)))
           ( \ (str: #IOI/data)
            -> ((((#IOI/putLine r) (#Maybe/@ #IOI/data)) str)
                (#Maybe/Nothing #IOI/data))))
         (((#IOI/getLine r) (#Maybe/@ #IOI/data))
          (#Maybe/Just #IOI/data))))))
\end{lstlisting}

Erlang Coinductive Bindings.

\begin{lstlisting}[mathescape=true]
copure() ->
    fun (_) -> fun (IO) -> IO end end.

cogetLine() ->
    fun(IO) -> fun(_) ->
        L = ch:list(io:get_line("> ")),
        ch:ap(IO,[L]) end end.

coputLine() ->
    fun (S) -> fun(IO) ->
        X = ch:unlist(S),
        io:put_chars(": "++X),
        case X of "0\n" -> list([]);
                      _ -> corec() end end end.

corec() ->
    ap('Morte':corecursive(),
        [copure(),cogetLine(),coputLine(),copure(),list([])]).
\end{lstlisting}

\begin{lstlisting}[mathescape=true]
> om_extract:extract("priv/normal/IOI").
ok
> Active: module loaded: {reloaded,'IOI'}
\end{lstlisting}

\begin{lstlisting}[mathescape=true]
> om:corec().
> 1
: 1
> 0
: 0
#Fun<List.3.113171260>
\end{lstlisting}

\subsubsection{I/O Type}

I/O Type Spec.

\begin{lstlisting}[mathescape=true]
-- IO/@
   \ (a : *)
-> \/ (IO : *)
-> \/ (GetLine_ : (#IO/data -> IO) -> IO)
-> \/ (PutLine_ : #IO/data -> IO -> IO)
-> \/ (Pure_ : a -> IO)
-> IO

-- IO/replicateM
   \ (n: #Nat/@)
-> \ (io: #IO/@ #Unit/@)
-> #Nat/fold n (#IO/@ #Unit/@)
               (#IO/[>>] io)
               (#IO/pure #Unit/@ #Unit/Make)
\end{lstlisting}

Guarded Recursion I/O Sample Program.

\begin{lstlisting}[mathescape=true]
-- Morte/recursive
((#IO/replicateM #Nat/Five)
 ((((#IO/[>>=] #IO/data) #Unit/@) #IO/getLine) #IO/putLine))
\end{lstlisting}

Erlang Inductive Bindings.

\begin{lstlisting}[mathescape=true]
pure() ->
    fun(IO) -> IO end.

getLine() ->
    fun(IO) -> fun(_) ->
        L = ch:list(io:get_line("> ")),
        ch:ap(IO,[L]) end end.

putLine() ->
    fun (S) -> fun(IO) ->
        io:put_chars(": "++ch:unlist(S)),
        ch:ap(IO,[S]) end end.

rec() ->
    ap('Morte':recursive(),
        [getLine(),putLine(),pure(),list([])]).
\end{lstlisting}


Here is example of Erlang/OTP shell running recursive example.

\begin{lstlisting}[mathescape=true]
> om:rec().
> 1
: 1
> 2
: 2
> 3
: 3
> 4
: 4
> 5
: 5
#Fun<List.28.113171260>
\end{lstlisting}

\section{Top Language with Inductive Types}

Despite we can encode inductive types in PTS, the best usage of inductive types
comes with recursors and fixpoint type that allow recursive typecheck, limiting
from the other side the normalization properties.

\paragraph{}
So called Calculus of Inductive Constructions is used as a top language on top of
PTS to reason about inductive types. Here we will show you a sketch of such
inductive language model which intendent to be a language extension to PTS system.

\paragraph{}
Exe is a general purpose functional language with $\Pi$ and $\Sigma$ types,
recursive algebraic types, higher order functions,
corecursion, and a free monad to encode effects. It compiles
to a small MLTT core of dependent type system with inductive types and equality.

\paragraph{}
It also has an $Id$ type (with its recursor) wich is useful for proving things,
$Let$ sytax sugar, case analysis over inductive types and also a $pi$-Calculus primitives:
$spawn$, $send$ and $receive$ which are native mappings to Erlang/OTP.

\subsection{BNF}

\begin{lstlisting}[mathescape=true]
    <> ::= #option
    [] ::= #list
     | ::= #sum
     1 ::= #unit
     I ::= #identifier

     U ::= Type < #nat >
     T ::= 1 | ( I : O ) T
     F ::= 1 | I : O = O , F
     B ::= 1 | [ | I [ I ] $\rightarrow$ O ]
     L ::= 1 | I T
     O ::= I | ( O ) |
           U | O $\rightarrow$ O
             | O O
             | fun ( I : O ) $\rightarrow$ O
             | fst O
             | snd O
             | id O O O
             | transport O O O O O
             | ( I : O ) * O
             | ( I : O ) $\rightarrow$ O
             | data   I T : O := T
             | record I T : O := T
             | let F in O
             | case O B
             | receive B
             | spawn O O
             | send O O
\end{lstlisting}

That is how we see the language informally.

However in real life we use grammar parser generators.
The Erlang version of parser encoded with OTP library {\bf yecc} which implements
LALR-1 grammar generator. This version resembles the model and slightly based on cubical
type checker by Mortberg et all.

\begin{lstlisting}
mod   -> 'module' id 'where' imp dec       : {module,'$2','$3','$4','$5'}.
imp   -> skip imp                          : '$2'.
imp   -> '$empty'                          : [].
imp   -> 'import' id imp                   : [{import,'$2'}|'$3'].
tele  -> '$empty'                          : [].
tele  -> '(' exp ':' exp ')' tele          : {tele,uncons('$2'),'$4','$6'}.
exp   -> app                               : '$1'.
exp   -> exp arrow exp                     : {arrow,'$1','$3'}.
exp   -> '(' exp ')'                       : '$2'.
exp   -> lam '(' exp ':' exp ')' arrow exp : {lam,uncons('$3'),'$5','$8'}.
exp   -> '(' exp ':' exp ')' arrow exp     : {pi,uncons('$2'),'$4','$7'}.
exp   -> '(' id ':' exp ')' '*' exp        : {sigma,'$2','$4','$7'}.
exp   -> id                                : '$1'.
app   -> exp exp                           : {app,'$1','$2'}.
dec   -> '$empty'                          : [].
dec   -> def skip dec                      : ['$1'|'$3'].
def   -> 'data' id tele '=' sum            : {data,'$2','$3','$5'}.
def   -> id tele ':' exp '=' exp           : {def,'$1','$2','$4','$6'}.
sum   -> '$empty'                          : [].
sum   -> id tele                           : {ctor,'$1','$2'}.
sum   -> id tele '|' sum                   : [{ctor,'$1','$2'}|'$4'].
\end{lstlisting}

\subsection{AST}

The model in Cubical and Coq of the Exe language is available at
infinity\footnote{https://github.com/groupoid/infinity/tree/master/priv}
repository of groupoid organization.

\begin{lstlisting}[mathescape=true]
 data tele (A: U)   = emp | tel (n: name) (b: A) (t: tele A)
 data branch (A: U) =        br (n: name) (args: list name) (term: A)
 data label (A: U)  =       lab (n: name) (t: tele A)
 data ind
    = star                        (n: nat)
    | var    (n: name)            (i: nat)
    | app              (f a: ind)
    | lambda (x: name) (d c: ind)
    | pi     (x: name) (d c: ind)
    | sigma  (n: name) (a b: ind)
    | arrow            (d c: ind)
    | pair             (a b: ind)
    | fst              (p:   ind)
    | snd              (p:   ind)
    | id               (a b: ind)
    | idpair           (a b: ind)
    | idelim           (a b c d e: ind)
    | data_  (n: name) (t: tele ind) (labels:   list (label ind))
    | case   (n: name) (t: ind)      (branches: list (branch ind))
    | ctor   (n: name)               (args:     list ind)
\end{lstlisting}

\subsection{Inductive Types}

\paragraph{}
There are two types of recursion: one is the least fixed point (as $F_A\ X = 1 + A\times X$
or $F_A\ X = A + X\times X$), in other words the recursion with a base (terminated with a bounded value),
lists and trees are examples of such recursive structures (so we call induction recursive sums);
and the second is the greatest fixed point or recursion without a base (as $F_A\ X = A\times X $) ---
such kind of recursion on infinite lists (codata, streams, coinductive types) we can call recursive products.\\

\subsection{Polynomial Functors}
Least fixed point trees are called well-founded trees. They encode polynomial functors.\\

\noindent Natural Numbers: $\mu\ X \rightarrow 1 + X$\\
List A: $\mu\ X \rightarrow 1 + A \times X$\\
Lambda calculus: $\mu\ X \rightarrow 1 + X \times X + X$\\
Stream: $\nu\ X \rightarrow A \times X$\\
Potentialy Infinite List A: $\nu\ X \rightarrow 1 + A \times X$\\
Finite Tree: $\mu\ X \rightarrow \mu\ Y \rightarrow 1 + X \times Y = \mu\ X = List\ X$\\

\paragraph{}
As we know there are several ways to appear for a variable in a recursive algebraic type.
Least fixpoint are known as an recursive expressions that have a base of recursion
Both recursive and corecursive datatypes could be encoded using Boehm-Berarducci encoding
as  non-recursive definitions of folds that include in identity signature all the
constructor components of (co)inductive type.

\subsection{Lists}
The data type of lists over a given set A can be represented as the initial algebra
$(\mu L_A, in)$ of the functor $L_A(X) = 1 + (A \times X)$. Denote $\mu L_A = List(A)$.
The constructor functions $nil: 1 \rightarrow List(A)$ and
$cons: A \times List(A) \rightarrow List(A)$ are defined by
$nil = in \circ inl$ and $cons = in \circ inr$, so $in = [nil,cons]$.
Given any two functions $c: 1 \rightarrow C$ and $h: A \times C \rightarrow C$,
the catamorphism $f = \llparenthesis [c,h] \rrparenthesis : List(A) \rightarrow C$
is the unique solution of the simultaneous equations:
\vspace{0.3cm}
$$
\begin{cases}
  f \circ nil  = c \\
  f \circ cons = h \circ (id \times f)
\end{cases}
$$

\paragraph{}
where $f = foldr(c,h)$. Having this the initial algebra is presented with functor
$\mu (1 + A \times X)$ and morphisms sum $[1 \rightarrow List(A), A \times List(A) \rightarrow List(A)]$
as catamorphism. Using this encoding the base library of List will have following form:

\vspace{0.5cm}
$$
\begin{cases}
foldr = \llparenthesis [ f \circ nil , h] \rrparenthesis, f \circ cons = h \circ (id \times f)\\
len = \llparenthesis [ zero, \lambda\ a\ n \rightarrow succ\ n ] \rrparenthesis \\
(++) = \lambda\ xs\ ys \rightarrow \llparenthesis [ \lambda (x) \rightarrow ys, cons ] \rrparenthesis (xs) \\
map = \lambda\ f \rightarrow \llparenthesis [ nil, cons \circ (f \times id)] \rrparenthesis
\end{cases}
$$

\begin{lstlisting}[mathescape=true]
             data list: (A: *) $\rightarrow$ * :=
                  (nil: list A)
                  (cons: A $\rightarrow$ list A $\rightarrow$ list A)
\end{lstlisting}

$$
\begin{cases}
list = \lambda\ ctor \rightarrow \lambda\ cons \rightarrow \lambda\ nil \rightarrow ctor\\
cons = \lambda\ x\ \rightarrow \lambda\ xs \rightarrow \lambda\ list \rightarrow \lambda\ cons \rightarrow\ \lambda\ nil \rightarrow cons\ x\ (xs\ list\ cons\ nil)\\
nil = \lambda\ list \rightarrow \lambda\ cons \rightarrow \lambda\ nil \rightarrow nil\\
\end{cases}
$$

\begin{lstlisting}[mathescape=true]
-- List/@
   \ (A : *)
-> \/ (List: *)
-> \/ (Cons: \/ (Head: A) -> \/ (Tail: List) -> List)
-> \/ (Nil: List)
-> List

-- List/Cons
   \ (A: *)
-> \ (Head: A)
-> \ (Tail:
      \/ (List: *)
   -> \/ (Cons:
         \/ (Head: A)
      -> \/ (Tail: List)
      -> List)
   -> \/ (Nil: List)
   -> List)
-> \ (List: *)
-> \ (Cons:
      \/ (Head: A)
   -> \/ (Tail: List)
   -> List)
-> \ (Nil: List)
-> Cons Head (Tail List Cons Nil)

-- List/Nil
   \ (A: *)
-> \ (List: *)
-> \ (Cons:
      \/ (Head: A)
   -> \/ (Tail: List)
   -> List)
-> \ (Nil: List)
-> Nil
\end{lstlisting}

\begin{lstlisting}[mathescape=true]
           record lists: (A B: *) :=
                  (len: list A $\rightarrow$ integer)
                  ((++): list A $\rightarrow$ list A $\rightarrow$ list A)
                  (map: (A $\rightarrow$ B) $\rightarrow$ (list A $\rightarrow$ list B))
                  (filter: (A $\rightarrow$ bool) $\rightarrow$ (list A $\rightarrow$ list A))
\end{lstlisting}
$$
\begin{cases}
len = foldr\ (\lambda\ x\ n \rightarrow succ\ n)\ 0\\
(++) = \lambda\ ys \rightarrow foldr\ cons\ ys\\
map = \lambda\ f \rightarrow foldr\ (\lambda x\ xs \rightarrow cons\ (f\ x)\ xs)\ nil\\
filter = \lambda\ p \rightarrow foldr\ (\lambda x\ xs \rightarrow if\ p\ x\ then\ cons\ x\ xs\ else\ xs)\ nil\\
foldl = \lambda\ f\ v\ xs = foldr\ (\lambda\ xg\rightarrow\ (\lambda \rightarrow g\ (f\ a\ x)))\ id\ xs\ v\\
\end{cases}
$$

\vspace{1cm}
\subsection{Normal Forms}

Here is example of List/map generic function.

\subsubsection*{Lists/map}
{\fontfamily{pcr}\selectfont
\vspace{0.5cm}
$\lambda$ (a: *) $\rightarrow$ $\lambda$ (b: *) $\rightarrow$ $\lambda$ (f: a $\rightarrow$ b) $\rightarrow$ $\lambda$ (xs: $\forall$ (List: *)
$\rightarrow$ $\forall$ (Cons: $\forall$ (head: a) $\rightarrow$ $\forall$ (tail: List) $\rightarrow$ List) $\rightarrow$ $\forall$ (Nil: List) $\rightarrow$ List)
$\rightarrow$ xs ($\forall$ (List: *) $\rightarrow$ $\forall$ (Cons: $\forall$ (head: b) $\rightarrow$ $\forall$ (tail: List) $\rightarrow$ List)
$\rightarrow$ $\forall$ (Nil: List) $\rightarrow$ List) ($\lambda$ (head: a) $\rightarrow$ $\lambda$ (tail: $\forall$ (List: *) $\rightarrow$
$\forall$ (Cons: $\forall$ (head: b) $\rightarrow$ $\forall$ (tail: List) $\rightarrow$ List) $\rightarrow$ $\forall$ (Nil: List)
$\rightarrow$ List) $\rightarrow$ $\lambda$ (List: *) $\rightarrow$ $\lambda$ (Cons: $\forall$ (head: b) $\rightarrow$ $\forall$
(tail: List) $\rightarrow$ List) $\rightarrow$ $\lambda$ (Nil: List) $\rightarrow$ Cons (f head) (tail List Cons Nil))
($\lambda$ (List: *) $\rightarrow$ $\lambda$ (Cons: $\forall$ (head: b) $\rightarrow$ $\forall$ (tail: List) $\rightarrow$
List) $\rightarrow$ $\lambda$ (Nil: List) $\rightarrow$ Nil)
}

\subsection{Prelude Base Library}

The base library is modelled in cubical type checker.

\begin{lstlisting}[mathescape=true]

             data Nat: Type :=
                  (Zero: Unit $\rightarrow$ Nat)
                  (Succ: Nat $\rightarrow$ Nat)

             data List (A: Type) : Type :=
                  (Nil: Unit $\rightarrow$ List A)
                  (Cons: A $\rightarrow$ List A $\rightarrow$ List A)

           record list: Type :=
                  (len: List A $\rightarrow$ integer)
                  ((++): List A $\rightarrow$ List A $\rightarrow$ List A)
                  (map: (A,B: Type) (A $\rightarrow$ B) $\rightarrow$ (List A $\rightarrow$ List B))
                  (filter: (A $\rightarrow$ bool) $\rightarrow$ (List A $\rightarrow$ List A))

           record String: List Nat := List.Nil

             data IO: Type :=
                  (getLine: (String $\rightarrow$ IO) $\rightarrow$ IO)
                  (putLint: String $\rightarrow$ IO)
                  (pure: () $\rightarrow$ IO)

           record IO: Type :=
                  (data: String)
                  ([>>=]: ...)

           record Morte: Type :=
                  (recursive: IO.replicateM Nat.Five
                              (IO.[>>=] IO.data Unit IO.getLine IO.putLine))

\end{lstlisting}

\subsection{Compiler Passes}
The underlying OM typechecker and compiler is a target language for EXE general purpose language.
The overall size of OM language with extractor to Erlang is 263 lines of code.

\begin{center}
\begin{tabular}{lll}
   TOKEN   & 54 LOC & Handcoded Tokenizer\\
   PARSER  & 81 LOC & Parser\\
   NORMAL  & 60 LOC & Term normalization and typechecking\\
   ERASE   & 36 LOC & Delete information about types\\
   EXTRACT & 32 LOC & Extract Erlang Code\\
\end{tabular}
\end{center}

We benchmarked the unrolling of inductive list type in Church
encoding extracted with OM with native erlang {\bf lists:foldl}.

\begin{lstlisting}[mathescape=true]
Pack/Unpack 1 000 000 Inductive Nat:   776407 us
Pack/Unpack 1 000 000 ErlangOTP List:  148084 us
Pack/Unpack 1 000 000 Inductive List: 1036461 us
\end{lstlisting}

\section{Acknowledgments}
I would like to thank to all people who inspired me for this work,
all contributors to Groupoid Infinity who helped me to avoid mistakes
in TeX and Erlang files.

\bibliographystyle{aipnum-cp}
\bibliography{om-aip}

\end{document}
