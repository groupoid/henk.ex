% copyright (c) 2015 Synrc Research Center

\documentclass[11pt,oneside]{article}

% Copyright (c) 2010 Synrc Research Center

\usepackage{ifthen}
\usepackage[english]{babel}
\usepackage{bussproofs}
\usepackage{tabstackengine}
\usepackage{graphicx}
\usepackage{cite}
\usepackage[usenames,dvipsnames]{color}
\usepackage[hmarginratio=2:3]{geometry}
\usepackage{inconsolata}
\usepackage{amssymb}
\usepackage{amsmath}
\usepackage[lBrack,rBrack,llparenthesis,rrparenthesis]{stmaryrd} % for comparison with 4 similar parenthesis symbols
\usepackage{accsupp}

\usepackage{mathtools}
\usepackage{hyperref}
\usepackage[utf8]{inputenc}
\usepackage[T1]{fontenc}
\usepackage{listingsutf8}
\usepackage{moreverb}
\usepackage{listings}
\usepackage[none]{hyphenat}
\usepackage{caption}
\usepackage{pgf}
\usepackage{tikz-cd}

\lstset{morekeywords={record,data,inductive,extend,enum,Type,Path,unit,Unit,Nat,List,let,in,spawn,send,receive,case,sigma,pi,fst,snd,option,nat,list,sum,identifier,lambda,arrow,star,var,app,remote,sh,sub,forall,norm,type,true,false,func,id,transport,eq,dep,h,fun}}

\hyphenation{framework nitrogen javascript facebook}

% include image for HeVeA and LaTeX

\makeatletter
\def\@seccntformat#1{\llap{\csname the#1\endcsname\quad}}
\makeatother

\newcommand{\includeimage}[2]
{\ifhevea
    {\imgsrc{#1}}
\else{
    \begin{figure}[h!]
    \centering
    \includegraphics[width=\textwidth]{#1}
    \caption{#2}
    \end{figure}}
\fi}

\lstset{
    backgroundcolor=\color{white},
    keywordstyle=\color{blue},
    inputencoding=utf8,
    basicstyle=\bf\ttfamily\footnotesize,
    columns=fixed}

\headsep = 0cm
\voffset = 0cm
\hoffset = 1cm
\topmargin = 0cm
\textwidth = 15cm
\textheight = 22cm
\footskip = 1.3cm
\parindent = 0cm

\hyphenpenalty=5000
  \tolerance=1000

\newcommand{\sign}[1]{%      
  \begin{tabular}[t]{@{}l@{}}
  \makebox[1.5in]{\dotfill}\\
  \strut#1\strut
  \end{tabular}%
}
\newcommand{\Date}{%
  \begin{tabular}[t]{@{}p{1.5in}@{}}
  \\[-1ex]
  \strut Date: \dotfill\strut
  \end{tabular}%
}

\addto\captionenglish{\renewcommand*{\bibname}{Literaturliste}}

\lstset{
 inputencoding=utf8,
extendedchars=true
}

\begin{document}

\thispagestyle{empty}
\begin{center}

\begin{minipage}[t]{2cm}
    \includegraphics[scale=0.4]{img/S}
\end{minipage}
\begin{minipage}[t]{12cm}
    \begin{flushright}
        \textsc{{\Large {\bf {\color{Blue}syn}{\color{OrangeRed}rc} research center s.r.o.}}}\\
        \textsc{Roháčova 141/18, Praha 3 13000, Czech Republic}\\
    \end{flushright}
\end{minipage}

\vspace{3cm}

    \vspace{3cm}   {\Large \bf Normal Form Representation of Dependent Inductive Types\\ \vspace{0.2cm}
                                in terms of Pure Type System\\}\par
    \vspace{0.3cm} {\Large Technical Article\par}
    \vspace{0.3cm} {\Large Maxim Sokhatsky, Synrc Research Center\par}
    \vspace{4cm}   {\Large Kyiv 2016}

\end{center}

\newpage
\vspace{2cm}
\tableofcontents
\newpage
\section{Introduction}

\vspace{1cm}

%\vspace{0.5cm}

\subsection{Verifiable Functional Compiler}

   \paragraph{}
   This article describes the frontend language {\bf Exe} along with its categorical semantic
   that compiles to a pure type system minimal core {\bf Om} --- the
   non-recursive subset of dependent type theory from which you can extract: LLVM code,
   untyped lambda code with erased type information, or even $System\ F_\omega$ programs.
   As a top-level task we expect to produce lean verifiable functional compiler pipeline,
   but as an milestone task we just produce concise working prototype in Erlang language.

\subsection{General Purpose Functional Language Exe}

   \paragraph{}
   General purpose function language with functors, lambdas on types, recursive algebraic types,
   higher order functions and embedded process calculus with corecursion, free monad for effects encoding.
   This language will be called Exe and dedicated to be high level general purpose functional
   programming language frontend to small core of dependent type system without recursion called Om.
   This language indended to be useful enough to encode KVS, N2O and BPE applications.

\subsection{Intermediate Language Om}

   \paragraph{}
   An intermediate Om language is based on Henk\cite{henk} languages described first
   by Erik Meyer and Simon Peyton Jones in 1997. Leter on in 2015 Morte impementation
   of Henk design appeared in Haskell, using Boem-Berrarducci encoding of non-recursive lamda terms.
   It is based only on $\pi$, $\lambda$ and $apply$ constructions, one axiom and four deduction rules.
   The design of Om language resemble Henk and Morte both design and implementation.
   This language indended to be small, conside, easy provable and clean and produce
   verifiable peace of code that can be distributed over the networks and compiled at target with
   safe linkage.

\subsection{Target Erlang VM and LLVM platforms}

   \paragraph{}
   This works expect to compile to limited target platforms. For now Erlang, Haskell and LLVM is awaiting.
   Erlang version is expected to be useful both on LING and BEAM Erlang virtual machines.

\newpage
\section{General Purpose Language}
\vspace{0.3cm}

   \subsection{Category Theory, Programs and Functions}
   Category theory is widely used as an instrument for mathematicians for software analisys.
   Category theory could be treated as an abstract algebra of functions. Let's define an Category
   formally: {\bf Category} consists of two lists: the one is morphisms (arrows) and the second is
   objects (domains and codomains of arrows) along with assoociative operation of composition and
   unit morphism that exists for all objects in category.

   \paragraph{}
   The formation axoims of objects and arrows are not given here and autopostulating yet. Formation axoims
   will be introduced during exponential definition. Objects $A$ and $B$ of an arrow $f: A \rightarrow B$
   are called {\bf domain} and {\bf codomain} respectively.

   \paragraph{}
   Intro axioms -- associativity of composition and left/right unit arrow compisitions show that
   categories are actually typed monoids, which consist of morphisms and operation of composition.
   There are many languages to show the semantic of categories such as commutative diagrams and string diagrams
   however here we define here in proof-theoretic manner:

\begingroup
\parbox[t][][l]{0.60\textwidth}{

\begin{prooftree}
\AxiomC{$\Gamma\ \vdash f: A \rightarrow B$ }
\AxiomC{$\Gamma\ \vdash g: B \rightarrow C$ }
\BinaryInfC{$\Gamma \vdash g \circ f : A \rightarrow C $}
\end{prooftree}

\begin{prooftree}
\AxiomC{$\Gamma \vdash f : B \rightarrow A$ }
\AxiomC{$\Gamma \vdash g : C \rightarrow B$ }
\AxiomC{$\Gamma \vdash h : D \rightarrow C$ }
\TrinaryInfC{$\Gamma \vdash (f \circ g) \circ h = f \circ (g \circ h) : D \rightarrow A $}
\end{prooftree}

}
\hspace{0.1cm}
\parbox[t][][r]{0.40\textwidth}{

\begin{prooftree}
\AxiomC{$$ }
\UnaryInfC{$\Gamma \vdash id_A : A \rightarrow A $}
\end{prooftree}

\begin{prooftree}
\AxiomC{$\Gamma\ \vdash f: A \rightarrow B$ }
\UnaryInfC{$\Gamma \vdash f \circ id_A = f : A \rightarrow B$}
\end{prooftree}

\begin{prooftree}
\AxiomC{$\Gamma\ \vdash f: A \rightarrow B$ }
\UnaryInfC{$\Gamma \vdash id_B \circ f = f : A \rightarrow B $}
\end{prooftree}

}
\endgroup

\paragraph{}
Composition shows an ability to connect the result space of the previous evaluation (codomain)
and the arguments space of the next evaluation (domain). Composition is fundamental property of morphisms
that allows us to chain evaluations.

\paragraph{}
\begin{tabular}{lll}
$1.$ & $A: *$\\
$2.$ & $A: *\ ,\ B: * \implies f: A \rightarrow B$\\
$3.$ & $f: B \rightarrow C\ ,\ g: A \rightarrow B \implies f \circ g : A \rightarrow C$\\
$4.$ & $(f \circ g) \circ h = f \circ (g \circ h)$\\
$5.$ & $A \implies id : A \rightarrow A$\\
$6.$ & $f \circ id = f$\\
$7.$ & $id \circ f = f$\\
\end{tabular}

\newpage
\subsection{Algebraic Types and Cartesian Categories}

After composition operation of construction of new objects with morphisms we introduce
operation of construction cartesian product of two objects $A$ and $B$ of a given
category along with morphism product $<f,g>$ with a common domain, that is needed
for full definition of cartesian product of $A \times B$.

\paragraph{}
This is an internal language of cartesian category, in which for all two selected objects there is an object
of cartesian product (sum) of two objects along with its $\bot$ terminal (or $\top$ coterminal) type.
Exe languages is always equiped with product and sum types.

\paragraph{}
Product has two eliminators $\pi$ with an common domain, which are also called projections of an product.
The sum has eliminators $i$ with an common codomain.
Note that eliminators $\pi$ and $i$ are isomorphic, that is $\pi \circ \sigma = \sigma \circ \pi = id$.

\begingroup
\parbox[t][][l]{0.40\textwidth}{

\begin{prooftree}
\AxiomC{$\Gamma\ x: A \times B$ }
\UnaryInfC{$\Gamma \vdash \pi_1\ : A \times B \rightarrow A$;
           $\Gamma \vdash \pi_2\ : A \times B \rightarrow B$}
\end{prooftree}

\begin{prooftree}
\AxiomC{$\Gamma \vdash\  a:A$ }
\AxiomC{$\Gamma \vdash\  b:B$ }
\BinaryInfC{$\Gamma \vdash\ (a,b) : A \times B$ }
\end{prooftree}

\begin{prooftree}
\AxiomC{}
\UnaryInfC{$\Gamma \vdash\ \top$ }
\end{prooftree}

}
\hspace{0.1cm}
\parbox[t][][r]{0.60\textwidth}{

\begin{prooftree}
\AxiomC{}
\UnaryInfC{$\Gamma \vdash\ \bot$ }
\end{prooftree}

\begin{prooftree}
\AxiomC{$\Gamma \vdash\  a:A$ }
\AxiomC{$\Gamma \vdash\  b:B$ }
\BinaryInfC{$\Gamma\vdash a\ |\ b : A \otimes B$}
\end{prooftree}

\begin{prooftree}
\AxiomC{$\Gamma\ x: A \otimes B$ }
\UnaryInfC{$\Gamma \vdash \i_1: A \rightarrow A \otimes B$;
           $\Gamma \vdash \i_2: B \rightarrow A \otimes B$}
\end{prooftree}

}
\endgroup

   \paragraph{}

   The $\bot$ type in Haskell is used as {\bf undefined} type (empty sum component presented in all types), that
   is why Hask category is not based on cartesian closed but CPO\cite{cpo}. The $\bot$ type has no values.
   The $\top$ type is known as unit type or zero tuple $()$ often
   used as an default argument for function with zero arguments.
   Also we include here an axiom of morphism product which is given during full definition
   of product using commutative diagram. This axiom is needed for applicative
   programming in categorical abstract machine. Also consider co-version of this
   axiom for $[f,g]: B+C \rightarrow A$ morphism sums.

\begin{prooftree}
\AxiomC{$\Gamma \vdash\ f:A \rightarrow B$ }
\AxiomC{$\Gamma \vdash\ g:A \rightarrow C$ }
\AxiomC{$\Gamma \vdash\ B \times C$ }
\TrinaryInfC{$\Gamma \vdash\ \langle f,g \rangle : A \rightarrow B \times C$ }
\end{prooftree}

\begin{center}
%$(f \circ g) \circ h = f \circ (g \circ h)$\\
%$f \circ id = f$\\
%$id \circ f = f$\\
$\pi_1 \circ \langle f, g \rangle = f$\\
$\pi_2 \circ \langle f, g \rangle = g$\\
$\langle f \circ \pi_1, f \circ \pi_2 \rangle = f$\\
$\langle f, g \rangle \circ h = \langle f \circ h, g \circ h \rangle$\\
$\langle \pi_1, \pi_2 \rangle = id$\\
\end{center}

\newpage
   \subsection{Exponential, $\lambda$-calculus and Cartesian Closed Categories}
   Being an internal language of cartesian closed category, lambda calculus except variables and constants
   provides two operations of abstraction and applications which defines complete evaluation language
   with higher order functions, recursion and corecursion, etc.

   \paragraph{}
   To explain functions from the categorical point of vew we need to define categorica exponential
   $f: A^B$, which are analogue to functions $f: A \rightarrow B$.
   As we already defined the products and terminals we could define an exponentials with three
   axioms of function construction, one eliminator of application with apply a function to its argument
   and axiom of currying the function of two arguments to function of one argument.

\begingroup
\parbox[t][][l]{0.40\textwidth}{

\begin{prooftree}
\AxiomC{$\Gamma  x:A \vdash M : B$}
\UnaryInfC{$\Gamma \vdash \lambda\ x\ .\ M : A \rightarrow B$}
\end{prooftree}

\begin{prooftree}
\AxiomC{$\Gamma\ f:A \rightarrow B$ }
\AxiomC{$\Gamma\ a:A$ }
\BinaryInfC{$\Gamma \vdash apply\ f\ a\ : (A \rightarrow B) \times A \rightarrow B$}
\end{prooftree}

\begin{prooftree}
\AxiomC{$\Gamma \vdash f: A \times B \rightarrow C$ }
\UnaryInfC{$\Gamma \vdash curry\ f : A \rightarrow (B \rightarrow C)$}
\end{prooftree}

}
\hspace{0.1cm}
\parbox[t][][r]{0.60\textwidth}{

\begin{center}
$apply \circ \langle (curry\ f) \circ \pi_1 , \pi_2 \rangle = f$\\
$curry\ apply \circ \langle g \circ \pi_1, \pi_2 \rangle) = g$\\
$apply \circ \langle curry\ f, g \rangle = f \circ \langle id , g\rangle$\\
$(curry\ f) \circ g = curry\ (f \circ \langle g \circ \pi_1,\pi_2\rangle)$\\
$curry\ apply = id$\\
\end{center}


}
\endgroup

\subsection{$\lambda$-language}

\begin{center}
Objects : $\bot\ |\ \rightarrow\ |\ \times$\\
Morphisms : $id\ |\ f \circ g\ |\ \langle f, g \rangle\ |\ apply\ |\ \lambda\ |\ curry$
\end{center}

  \subsection{Functors, $\Lambda$-calculus}

  Functor comes as a notion of morphisms in categories whose objects are categories.
  Functors preserve compositions of arrows and identities, otherwise it would
  be impossible to deal with categories. One level up is notion of morphism between categories whose
  objects are Functors, such morphisms are called natural transformations. Here we need
  only functor definition which is needed as general type declarations.

\begin{prooftree}
\AxiomC{$\Gamma \vdash f\ :\ A \rightarrow B$}
\UnaryInfC{$\Gamma \vdash F\ f\ :\ (A \rightarrow B) \rightarrow (F\ a \rightarrow F\ b)$}
\end{prooftree}

\begin{prooftree}
\AxiomC{$\Gamma \vdash id_A\ :A \rightarrow A$}
\UnaryInfC{$\Gamma \vdash F\ id_A\ =\ id_{F A}\ :\ F\ A \rightarrow F\ A$}
\end{prooftree}

\begin{prooftree}
\AxiomC{$\Gamma \vdash f\ :B \rightarrow C,\ g : A \rightarrow B$}
\UnaryInfC{$\Gamma \vdash F\ f \circ F\ g\ =\ F (f \circ g)\ :\ F\ A \rightarrow F\ C$}
\end{prooftree}

   We start thinking of functors on dealing with typed theories, because functors usually
   could be seen as higher order type con

\newpage




\newpage
   \subsection{Algebras}

   F-Algebras gives us a categorical understanding recursive types.
   Let $F : C \rightarrow C$ be an endofunctor on category $C$.
   An F-algebra is a pair $(C, \phi)$, where C is an object and $\phi\ : F\ C \rightarrow C$
   an arrow in the category C. The object C is the carrier and the functor
   F is the signature of the algebra. Reversing arrows gives us F-Coalgebra.

\vspace{1cm}

\begin{center}
\begin{tabular}{lcl}
\begin{tikzcd}
  F\ C \arrow{d}[left]{F\ f} \arrow{r}{\varphi} & C \arrow{d}{f} \\
  F\ D \arrow{r}{\psi} & D \end{tikzcd} & & \begin{tikzcd}
  C \arrow{d}[left]{f} \arrow{r}{\varphi} & F\ C \arrow{d}{F\ f} \\
  D \arrow{r}{\psi} & F\ D \end{tikzcd} \\
  \ & \  &\  \\
  $f \circ \varphi = \psi \circ F\ f$ & & $\psi \circ f =  F\ f \circ \varphi$ \\
\end{tabular}
\end{center}

  \subsection{Initial Algebras}

  A F-algebra $(\mu F, in)$ is the initial F-algebra if for any F-algebra $(C, \varphi)$
  there exists a unique arrow $\llparenthesis \varphi \rrparenthesis : \mu F \rightarrow C$ where $f = \llparenthesis \varphi \rrparenthesis$
  and is called catamorphism. Similar a F-coalgebra $(\nu F, out)$ is the terminal
  F-coalgebra if for any F-coalgebra $(C, \varphi)$ there exists unique arrow
  $\llbracket \varphi \rrbracket : C \rightarrow \nu F$ where $f = 
  \llbracket \varphi \rrbracket$

\begin{center}
\begin{tabular}{lcl}
\begin{tikzcd}
  F\ \mu F \arrow{d}[left]{F\ \llparenthesis \varphi \rrparenthesis} \arrow{r}{in} & \mu F \arrow{d}{\llparenthesis \varphi \rrparenthesis} \\
  F C \arrow{r}{\varphi} & C \end{tikzcd} & & \begin{tikzcd}
  C \arrow{d}[left]{ \llbracket \varphi \rrbracket} \arrow{r}{\phi} & F\ C\arrow{d}{F\ \llbracket \varphi \rrbracket} \\
  \nu F \arrow{r}{out} & F \nu F\end{tikzcd} \\
  \ & \  &\  \\
  $f \circ in = \varphi \circ F\ f \equiv f = \llparenthesis \varphi \rrparenthesis$& &
  $out \circ f = F\ f \circ \varphi \equiv f = \llbracket \varphi \rrbracket$ \\
\end{tabular}
\end{center}

   \subsection{Recursive Types}

  As was shown by Wadler\cite{recursive} we could deal with recusrive equations having
  three axioms: one $fix: (A \rightarrow A) \rightarrow A$ fixedpoint axiom,
  and axioms $in: F\ T \rightarrow T$ and $out: T \rightarrow F\ T$ of recursion direction. We need to
  define fixed point as axiom because we can't define recursive axioms. This axioms
  also needs functor axiom defined earlier.


\begingroup
\parbox[t][][l]{0.40\textwidth}{

\begin{prooftree}
\AxiomC{$\Gamma  \vdash M : F\ (\mu\ F)$}
\UnaryInfC{$\Gamma \vdash in_{\mu F}\ M : \mu\ F$}
\end{prooftree}

\begin{prooftree}
\AxiomC{$\Gamma  \vdash M : \mu F$}
\UnaryInfC{$\Gamma \vdash out_{\mu F}\ M : F\ (\mu\ F)$}
\end{prooftree}

}
\hspace{0.1cm}
\parbox[t][][r]{0.60\textwidth}{


\begin{prooftree}
\AxiomC{$\Gamma  \vdash M : A \rightarrow A$}
\UnaryInfC{$\Gamma \vdash fix\ M : A$}
\end{prooftree}

}
\endgroup

\newpage
   \subsection{Inductive Types}

  \paragraph{}
  There is two types of recursion: one is least fixed point (as $F_A\ X = 1 + A\times X$ or $F_A\ X = A + X\times X$),
  in other words the recursion with a base (terminated with a bounded value), lists are trees are
  examples of such recursive structures (so we call induction recursive sums); and the second
  is greatest fixed point or recursion withour base (as $F_A\ X = A\times X $) --- such kind of
  recursion on infinite lists (codata, streams, coinductive types) we can call recursive products.\\
\\
  Natural Numbers: $\mu\ X \rightarrow 1 + X$ \\
  List A: $\mu\ X \rightarrow 1 + A \times X$ \\
  Lambda calculus: $\mu\ X \rightarrow 1 + X \times X + X$ \\
  Stream: $\nu\ X \rightarrow A \times X$ \\
  Potentialy Infinite List A: $\nu\ X \rightarrow 1 + A \times X$ \\
  Finite Tree: $\mu\ X \rightarrow \mu\ Y \rightarrow 1 + X \times Y = \mu\ X = List\ X$ \\

  \paragraph{}
  As we know there are several ways to appear for variable in recursive algebraic type.
  Least fixpoint are known as an recursive expressions that have a base of recursion
  Both recursive and corecursive datatypes could be encoded using Boem-Berarducci encoding
  as an non-recursive definitions of folds that include in indentity signature all the
  constructor components of (co)inductive type.

  \subsection{Peano Numbers}
  Pointer Unary System is a category C with terminal object
  and a carrier X having morphism $[zero: 1_C \rightarrow X,succ: X \rightarrow X]$.
  The initial object of C is called Natural Number Objects and models Peano axiom set.

\newpage
  \subsection{Lists}
  The data type of lists over a given set A can be represented as the initial algebra
  $(\mu L_A, in)$ of the functor $L_A(X) = 1 + (A \times X)$. Denote $\mu L_A = List(A)$.
  The constructor functions $nil: 1 \rightarrow List(A)$ and
  $cons: A \times List(A) \rightarrow List(A)$ are defined by
  $nil = in \circ inl$ and $cons = in \circ inr$, so $in = [nil,cons]$.
  Given any two functions $c: 1 \rightarrow C$ and $h: A \times C \rightarrow C$,
  the catamorphism $f = \llparenthesis [c,h] \rrparenthesis : List(A) \rightarrow C$
  is the unique solution of the equation system:
\vspace{0.3cm}
$$
\begin{cases}
  f \circ nil  = c \\
  f \circ cons = h \circ (id \times f)
\end{cases}
$$

\paragraph{}
  where $f = foldr(c,h)$. Having this the initial algebra is presented with functor
  $\mu (1 + A \times X)$ and morphisms sum $[1 \rightarrow List(A), A \times List(A) \rightarrow List(A)]$
  as catamorphism. Using this encdoding the base library of List will have following form:

\vspace{0.5cm}
$$
\begin{cases}
 foldr = \llparenthesis [ f \circ nil , h] \rrparenthesis, f \circ cons = h \circ (id \times f)\\
 len = \llparenthesis [ zero, \lambda\ a\ n \rightarrow succ\ n ] \rrparenthesis \\
 (++) = \lambda\ xs\ ys \rightarrow \llparenthesis [ \lambda (x) \rightarrow ys, cons ] \rrparenthesis (xs) \\
 map = \lambda\ f \rightarrow \llparenthesis [ nil, cons \circ (f \times id)] \rrparenthesis
\end{cases}
$$

\vspace{0.2cm}
\subsection{Inductive Encoding and List Module}
\vspace{0.4cm}

\begin{lstlisting}[mathescape=true]
        inductive list: (A: *) $\rightarrow$ * :=
                  (nil: list A)
                  (cons: A $\rightarrow$ list A $\rightarrow$ list A)
\end{lstlisting}
$$
\begin{cases}
list = \lambda\ ctor \rightarrow \lambda\ cons \rightarrow \lambda\ nil \rightarrow ctor\\
cons = \lambda\ x\ \rightarrow \lambda\ xs \rightarrow \lambda\ list \rightarrow \lambda\ cons \rightarrow\ \lambda\ nil \rightarrow cons\ x\ (xs\ list\ cons\ nil)\\
nil = \lambda\ list \rightarrow \lambda\ cons \rightarrow \lambda\ nil \rightarrow nil\\
\end{cases}
$$
\\
\begin{lstlisting}[mathescape=true]
           record lists: (A B: *) :=
                  (len: list A $\rightarrow$ integer)
                  ((++): list A $\rightarrow$ list A $\rightarrow$ list A)
                  (map: (A $\rightarrow$ B) $\rightarrow$ (list A $\rightarrow$ list B))
                  (filter: (A $\rightarrow$ bool) $\rightarrow$ (list A $\rightarrow$ list A))
\end{lstlisting}
$$
\begin{cases}
len = foldr\ (\lambda\ x\ n \rightarrow succ\ n)\ 0\\
(++) = \lambda\ ys \rightarrow foldr\ cons\ ys\\
map = \lambda\ f \rightarrow foldr\ (\lambda x\ xs \rightarrow cons\ (f\ x)\ xs)\ nil\\
filter = \lambda\ p \rightarrow foldr\ (\lambda x\ xs \rightarrow if\ p\ x\ then\ cons\ x\ xs\ else\ xs)\ nil\\
foldl = \lambda\ f\ v\ xs = foldr\ (\lambda\ xg\rightarrow\ (\lambda \rightarrow g\ (f\ a\ x)))\ id\ xs\ v\\
\end{cases}
$$


% \newpage
%   \subsection{Process Calculus}
%   Thr $\pi$-calculus of processes by Robert Milner is basic formalism for distributed
%   computational theory and its implementations. From origin times of CSP developed by Hoare,
%   Milner significanly evolved the theory and adopted it to contemporary telecommunication requirements,
%   such as hangovers in mobile networks.

%   \subsection{Model}

%\begin{lstlisting}

   
%\end{lstlisting}

%  \subsection*{Core Axioms}
%  We announce process and a fundamental function type of a special signature with protocol $\Sigma$ and
%  state $X$.

%\begin{prooftree}
%\AxiomC{$\Gamma\ \vdash E, \Sigma, X$ }
%\AxiomC{$\Gamma\ \vdash action : \Sigma \times X \rightarrow \Sigma \times X$ }
%\BinaryInfC{$\Gamma \vdash {spawn}\ action : \pi_(\Sigma,X) $}
%\end{prooftree}

%\begin{prooftree}
%\AxiomC{$\Gamma\ \vdash pid : \pi_(\Sigma,X)$ }
%\AxiomC{$\Gamma\ \vdash msg : \Sigma$ }
%\BinaryInfC{$\Gamma \vdash join\ msg\ pid : \Sigma \times \pi_\Sigma \xrightarrow{\bullet} \Sigma$;
%            $\Gamma \vdash send\ msg\ pid : \Sigma \times \pi_\Sigma \rightarrow \Sigma$}
%\end{prooftree}

%\begin{prooftree}
%\AxiomC{$\Gamma\ \vdash L : A + B, R : X + Y$ }
%\AxiomC{$\Gamma\ \vdash M : A \rightarrow X, N : B \rightarrow Y$ }
%\BinaryInfC{$\Gamma \vdash receive\ L\ M\ N : L \xrightarrow{\bullet} R$}
%\end{prooftree}

%\paragraph{}

%   \subsection*{Process Algebra}


% \begin{center}
% \begin{tabular}{lcl}
% $\oplus$   &:& $\pi \parallel \pi$\\
% $\otimes$  &:& $\pi \mid \pi$\\
% \end{tabular}
% \end{center}

\newpage
   \subsection{Intuitionistic Type Theory}

   \paragraph{}
   Using Build Yourself Type Teory approach sooner or later
   you should decide the pallete of inductive structures. Such in Coq
   abstract algebra framework was built upon polymorphic records\cite{coqalg} rather
   that type classes engine like it is used in Agda and Idris. However
   Idris still lack of polymorphic records and coinductive types. Lean also
   lack of coinductive structures but has powerful non-recursive polymorphic records
   which are used in Lean HoTT library.

   \paragraph{}
   As was show by Spephan Kaes\cite{kaes}, one of strategy of type clasess engine
   implementataion is using polymorphic structures which allows us do deal with
   persistent structures on a low theoretical level. Moreover this style of coding
   is completly compatible with Erlang records which are used to model KVS and N2O hierarchies.

  \subsection{Logic and Quantification}

  Depended product is generalisation of exponential or function definition, when domain of morphism is depended on codomain.
  Depended sum is generalisation of cartesian product when one element depends on anotehr.
  Depended product is noted as $\forall$ and depended sum is noted as $\exists$:

\begingroup
\parbox[t][][l]{0.40\textwidth}{

\begin{prooftree}
\AxiomC{$\Gamma\ x: A \vdash B$ }
\AxiomC{$\Gamma\ \vdash A$ }
\BinaryInfC{$\Gamma\ \vdash \Pi (x : A) B $}
\end{prooftree}

\begin{prooftree}
\AxiomC{$\Gamma\ x: A \vdash B$ }
\AxiomC{$\Gamma\ \vdash A$ }
\BinaryInfC{$\Gamma\ \vdash \Sigma (x : A) B $}
\end{prooftree}

}
\hspace{0.1cm}
\parbox[t][][r]{0.60\textwidth}{

\begin{prooftree}
\AxiomC{$\Gamma\ \vdash a : A$ }
\AxiomC{$\Gamma\ x : A \vdash B$ }
\AxiomC{$\Gamma\ b : B (x=a)$ }
\TrinaryInfC{$\Gamma\ \vdash (a,b) : \Pi (x : A) B $}
\end{prooftree}


\begin{prooftree}
\AxiomC{$\Gamma\ \vdash a : A$ }
\AxiomC{$\Gamma\ x : A \vdash B$ }
\AxiomC{$\Gamma\ b : B (x=a)$ }
\TrinaryInfC{$\Gamma\ \vdash (a,b) : \Sigma (x : A) B $}
\end{prooftree}

}
\endgroup

\begingroup
\parbox[t][][l]{0.40\textwidth}{

\begin{prooftree}
\AxiomC{$\Gamma\ \vdash x: A$ }
\AxiomC{$\Gamma\ \vdash x': A$ }
\BinaryInfC{$\Gamma\ \vdash Id_A (x,x')$}
\end{prooftree}

}
\hspace{0.1cm}
\parbox[t][][r]{0.60\textwidth}{

}\endgroup


\begin{center}
\begin{tabular}{lll}
  reflexivity     &:& $Id_A(a,a)$ \\
  substitution    &:& $Id_A(a,a') \rightarrow B(x=a) \rightarrow B(x=a')$ \\
  symmetry        &:& $Id_A(a,b) \rightarrow Id_A(b,a)$  \\
  transitivity    &:& $Id_A(a,b) \rightarrow Id_A(b,c) \rightarrow Id_A(a,c)$ \\
  congruence      &:& $(f: A \rightarrow B) \rightarrow Id_A(x,x') \rightarrow Id_B(f(x),f(x'))$ \\
\end{tabular}
\end{center}

\newpage

  \section{Intermediate Language}
\vspace{1cm}

   \subsection{Om Language Definition}
\vspace{0.5cm}
   Om resemble both the Henk theory of pure type system and $\lambda C$ calculus of constructions
   and Morte Core Specification

\begin{lstlisting}
    EXPR :=                     EXPR             EXPR 
          | "\"   "(" LABEL ":" EXPR ")" "arrow" EXPR 
          | "\/"  "(" LABEL ":" EXPR ")" "arrow" EXPR 
          |                     EXPR     "arrow" EXPR 
          |           LABEL                           
          | "*"                                       
          | "[]"                                      
          |       "("           EXPR ")"              
\end{lstlisting}

During forward pass we stack applications (except typevars), then
on reaching close paren ")" we perform backward pass and stack arrows,
until neaarest unstacked open paren "(" appeared (then we just return
control to the forward pass).

\paragraph{}
We need to preserve applies to typevars as they should
be processes lately on rewind pass, so we have just typevars bypassing rule.
On the rewind pass we stack lambdas by matching arrow/apply signatures
where typevar(x) is an introduction of variable "x" to the Gamma context.

\begin{center}
                   $apply: (A \rightarrow B) \times A \rightarrow B$ \\
                  $lambda: arrow\ (app\ (typevar\ x)\ ,\ A)\ B$ \\
\end{center}


   \subsection{Abstract Syntax Tree}

\begin{center}
\begin{tabular}{lcl}
      $E$ & :=& $K$ \\
          & | & $x$ \\
          & | & $E E$ \\
          & | & $\lambda (x: E) \rightarrow E$ \\
          & | & $\Pi (x: E) \rightarrow E$ \\
\end{tabular}
\end{center}

\newpage

  \subsection{Exe Prelude}
\vspace{1cm}
\paragraph{}

\begin{lstlisting}[mathescape=true]
             enum unit: * :=
                  (make: () $\rightarrow$ unit)

             enum bool: * :=
                  (true: () $\rightarrow$ bool)
                  (false: () $\rightarrow$ bool)

             enum option: (A:*) $\rightarrow$ * :=
                  (none: () $\rightarrow$ option A)
                  (some: A $\rightarrow$ option A)

             enum product: (A:*) $\rightarrow$ (B:*) $\rightarrow$ * :=
                  (make: (a:A) $\rightarrow$ (b:B) $\rightarrow$ prod A B)

             enum sum: (A:*) $\rightarrow$ (B:*) $\rightarrow$ * :=
                  (make: (a:A) $\rightarrow$ (b:B) -> sum A B)

             enum list: (A:*) $\rightarrow$ * :=
                  (nil: () $\rightarrow$ list A)
                  (cons: A $\rightarrow$ list A $\rightarrow$ list A)

           record lists: (A B: *) :=
                  (len: list A $\rightarrow$ integer)
                  ((++): list A $\rightarrow$ list A $\rightarrow$ list A)
                  (map: (A $\rightarrow$ B) $\rightarrow$ (list A $\rightarrow$ list B))
                  (filter: (A $\rightarrow$ bool) $\rightarrow$ (list A $\rightarrow$ list A))

             enum eq: (A:*) $\rightarrow$ A $\rightarrow$ A $\rightarrow$ * :=
                  (refl: (x:A) $\rightarrow$ eq A x x)

             enum exists: (A:*) $\rightarrow$ (A $\rightarrow$ *) $\rightarrow$ * :=
                  (exists-intro: (P: A $\rightarrow$ *) $\rightarrow$ (x:A) $\rightarrow$ P x $\rightarrow$ exists A P)

           record pure (P: * $\rightarrow$ *) (A: *) :=
                  (return: P A)

           record functor (F: * $\rightarrow$ *) (A B: *) :=
                  (fmap: (A $\rightarrow$ B) $\rightarrow$ F A $\rightarrow$ F B)

           record applicative (F: * $\rightarrow$ *) (A B: *)
           extend pure F A,
                  functor F A B :=
                  (ap: F (A $\rightarrow$ B) $\rightarrow$ F A $\rightarrow$ F B)

           record monad (F: * $\rightarrow$ *) (A B: *)
           extend pure F A,
                  functor F A B :=
                  (join: F (F A) $\rightarrow$ F B)

\end{lstlisting}

\newpage
\subsection*{Om Normal Depended Forms}
\vspace{1cm}
\subsubsection*{Lists/Map}
{\fontfamily{pcr}\selectfont
\vspace{0.5cm}
$\lambda$ (a: *) $\rightarrow$ $\lambda$ (b: *) $\rightarrow$ $\lambda$ (f: a $\rightarrow$ b) $\rightarrow$ $\lambda$ (xs: $\forall$ (List: *)
$\rightarrow$ $\forall$ (Cons: $\forall$ (head: a) $\rightarrow$ $\forall$ (tail: List) $\rightarrow$ List) $\rightarrow$ $\forall$ (Nil: List) $\rightarrow$ List)
$\rightarrow$ xs ($\forall$ (List: *) $\rightarrow$ $\forall$ (Cons: $\forall$ (head: b) $\rightarrow$ $\forall$ (tail: List) $\rightarrow$ List)
$\rightarrow$ $\forall$ (Nil: List) $\rightarrow$ List) ($\lambda$ (head: a) $\rightarrow$ $\lambda$ (tail: $\forall$ (List: *) $\rightarrow$
$\forall$ (Cons: $\forall$ (head: b) $\rightarrow$ $\forall$ (tail: List) $\rightarrow$ List) $\rightarrow$ $\forall$ (Nil: List)
$\rightarrow$ List) $\rightarrow$ $\lambda$ (List: *) $\rightarrow$ $\lambda$ (Cons: $\forall$ (head: b) $\rightarrow$ $\forall$
(tail: List) $\rightarrow$ List) $\rightarrow$ $\lambda$ (Nil: List) $\rightarrow$ Cons (f head) (tail List Cons Nil))
($\lambda$ (List: *) $\rightarrow$ $\lambda$ (Cons: $\forall$ (head: b) $\rightarrow$ $\forall$ (tail: List) $\rightarrow$
List) $\rightarrow$ $\lambda$ (Nil: List) $\rightarrow$ Nil)
}

\subsubsection*{Vector}
{\fontfamily{pcr}\selectfont
\vspace{0.5cm}
$\lambda$ (n : $\forall$ (Nat : *) $\rightarrow$ $\forall$ (Succ : Nat $\rightarrow$ Nat) $\rightarrow$ $\forall$ (Zero : Nat) $\rightarrow$ Nat) $\rightarrow$ $\lambda$ (a : *) $\rightarrow$ $\forall$ (Vector : ($\forall$ (Nat : *) $\rightarrow$ $\forall$ (Succ : Nat $\rightarrow$ Nat) $\rightarrow$ $\forall$ (Zero : Nat) $\rightarrow$ Nat) $\rightarrow$ * $\rightarrow$ *) $\rightarrow$ $\forall$ (Cons : $\forall$ (m : $\forall$ (Nat : *) $\rightarrow$ $\forall$ (Succ : Nat $\rightarrow$ Nat) $\rightarrow$ $\forall$ (Zero : Nat) $\rightarrow$ Nat) $\rightarrow$ $\forall$ (b : *) $\rightarrow$ b $\rightarrow$ Vector m b $\rightarrow$ Vector ($\lambda$ (Nat : *) $\rightarrow$ $\lambda$ (Succ : Nat $\rightarrow$ Nat) $\rightarrow$ $\lambda$ (Zero : Nat) $\rightarrow$ Succ (m Nat Succ Zero)) b) $\rightarrow$ $\forall$ (Nil : $\forall$ (b : *) $\rightarrow$ Vector ($\lambda$ (Nat : *) $\rightarrow$ $\lambda$ (Succ : Nat $\rightarrow$ Nat) $\rightarrow$ $\lambda$ (Zero : Nat) $\rightarrow$ Zero) b) $\rightarrow$ Vector n a
}

\begin{thebibliography}{9}

\bibitem{henk0}      Henk Barendregt \textit{The Lambda Calculus. Its syntax and semantics} 1981
\bibitem{henk1}      Henk Barendregt \textit{Lambda Calculus With Types} 2010
\bibitem{henk}       Erik Meijer, Symon Peyton Jones \textit{Henk: a typed intermediate language} 1984
% \bibitem{baastad}    P.Wadler \textit{Monads for functional programming}
\bibitem{lof}        Per Martin-Löf \textit{Intuitionistic Type Theory} 1984

\bibitem{curien1}    Pierre-Louis Curien \textit{Category theory: a programming language-oriented introduction}
% \bibitem{awodey}     S.Awodey \textit{Category Theory} 2010

% Inductive Types Section

\bibitem{vene}       Varmo Vene \textit{Categorical programming with (co)inductive types} 2000
\bibitem{pfenning}   Frank Pfenning \textit{Inductively defined types in the Calculus of Constructions} 1989
% \bibitem{recursive}  P.Wadler \textit{Recursive types for free} 2014 % http://homepages.inf.ed.ac.uk/wadler/papers/free-rectypes/free-rectypes.txt
% \bibitem{debjer}       P.Dybjer \textit{Inductive Famalies} 1997
% \bibitem{debjer2}      P.Dybjer \textit{Representing Inductively Defined Sets by Wellorderings in Martin Löf's Type Theory} 1996
% \bibitem{basold}       H.Basold,H.Geuvers \textit{Dependent Inductive and Coinductive Types are Fibrational Dialgebras} 2015
% \bibitem{jay}          B.Jay \textit{A semantics for shape}
% \bibitem{gambino}      N.Gambino,M.Hyland \textit{Wellfounded Trees and Dependent Polynomial Functors} 1995
% \bibitem{container}    M.Abbott,T.Altenkirch,N.Ghani, \textit{Categories of Containers} 2005

% \bibitem{baer}       A.Baer \textit{Programming with Algebraic Effects and Handlers} 2012
% \bibitem{lawvere}    William Lawvere. \textit{Conceptual Mathematics} 1997.
% \bibitem{comm}       Robin Milner. \textit{A Calculus of Communicating Systems.} 1986.
% \bibitem{commpi}     Robin Milner. \textit{Communicating and Mobile Systems: The $\pi$-calculus.} 1999.
% \bibitem{polypi}     Robin Milner. \textit{The Polyadic $\pi$-Calculus: A Tutorial.} 1993.
% \bibitem{coqhuet}    T.Coquand, G.Huet \textit{The Calculus of Constructions.} 1988
% \bibitem{chipvm}     A.Chlipala \textit{Certified Programming with Dependent Types} 2015
% \bibitem{idris}      E.Brady \textit{Programming in IDRIS: A Tutorial} 2015
% \bibitem{mcbrideapp} C.McBride, R.Patterson \textit{Applicative programming with effects} 2002
% \bibitem{andjelko}   S.Andjelkovic \textit{A family of universes for generic programming} 2011
% \bibitem{alacarte}   W.Swierstra \textit{Data types a` la carte} 2011
\bibitem{catlogic}     Bart Jacobs \textit{Categorical Logic and Type Theory} 1999
% \bibitem{pointfree}  A.Cuhna,J.Pinto,J.Proenca \textit{A Framework for Point-Free Program Transformation}
% \bibitem{kaes}       S.Kaes \textit{http://tuprints.ulb.tu-darmstadt.de/epda/000544/diss.pdf} 2005
% \bibitem{cpo}        NA.Danielsson,J.Hughes,J.Gibbons \textit{Fast and Loose Reasoning is Morally Correct} 2006

\end{thebibliography}
\newpage

%
%\begin{lstlisting}[mathescape=true]
%> om:show("priv/List/@").
%
%=INFO REPORT==== 22-Dec-2015::10:01:10 ===
%"priv/List/@"
%  $\lambda$ (a: *)
%$\rightarrow$ $\forall$ (List: *)
%$\rightarrow$ $\forall$ (Cons:
%    $\forall$ (head: a)
%  $\rightarrow$ $\forall$ (tail: List)
%  $\rightarrow$ List)
%$\rightarrow$ $\forall$ (Nil: List)
%$\rightarrow$ List

%{[],
% [{lambda,{{arg,a},
%           {const,star},
%           {pi,{{arg,'List'},
%                {const,star},
%                {pi,{{arg,'Cons'},
%                     {pi,{{arg,head},
%                          {var,{a,0}},
%                          {pi,{{arg,tail},
%                               {var,{'List',0}},
%                               {var,{'List',0}}}}}},
%                     {pi,{{arg,'Nil'},
%                          {var,{'List',0}},
%                          {var,{'List',0}}}}}}}}}}]}
%\end{lstlisting}
\end{document}
