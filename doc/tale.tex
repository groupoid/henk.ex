% copyright (c) 2016 Groupoid Infinity
\documentclass[11pt,oneside]{article}

% Copyright (c) 2010 Synrc Research Center

\usepackage{ifthen}
\usepackage[english]{babel}
\usepackage{bussproofs}
\usepackage{tabstackengine}
\usepackage{graphicx}
\usepackage{cite}
\usepackage[usenames,dvipsnames]{color}
\usepackage[hmarginratio=2:3]{geometry}
\usepackage{inconsolata}
\usepackage{amssymb}
\usepackage{amsmath}
\usepackage[lBrack,rBrack,llparenthesis,rrparenthesis]{stmaryrd} % for comparison with 4 similar parenthesis symbols
\usepackage{accsupp}

\usepackage{mathtools}
\usepackage{hyperref}
\usepackage[utf8]{inputenc}
\usepackage[T1]{fontenc}
\usepackage{listingsutf8}
\usepackage{moreverb}
\usepackage{listings}
\usepackage[none]{hyphenat}
\usepackage{caption}
\usepackage{pgf}
\usepackage{tikz-cd}

\lstset{morekeywords={record,data,inductive,extend,enum,Type,Path,unit,Unit,Nat,List,let,in,spawn,send,receive,case,sigma,pi,fst,snd,option,nat,list,sum,identifier,lambda,arrow,star,var,app,remote,sh,sub,forall,norm,type,true,false,func,id,transport,eq,dep,h,fun}}

\hyphenation{framework nitrogen javascript facebook}

% include image for HeVeA and LaTeX

\makeatletter
\def\@seccntformat#1{\llap{\csname the#1\endcsname\quad}}
\makeatother

\newcommand{\includeimage}[2]
{\ifhevea
    {\imgsrc{#1}}
\else{
    \begin{figure}[h!]
    \centering
    \includegraphics[width=\textwidth]{#1}
    \caption{#2}
    \end{figure}}
\fi}

\lstset{
    backgroundcolor=\color{white},
    keywordstyle=\color{blue},
    inputencoding=utf8,
    basicstyle=\bf\ttfamily\footnotesize,
    columns=fixed}

\headsep = 0cm
\voffset = 0cm
\hoffset = 1cm
\topmargin = 0cm
\textwidth = 15cm
\textheight = 22cm
\footskip = 1.3cm
\parindent = 0cm

\hyphenpenalty=5000
  \tolerance=1000

\newcommand{\sign}[1]{%      
  \begin{tabular}[t]{@{}l@{}}
  \makebox[1.5in]{\dotfill}\\
  \strut#1\strut
  \end{tabular}%
}
\newcommand{\Date}{%
  \begin{tabular}[t]{@{}p{1.5in}@{}}
  \\[-1ex]
  \strut Date: \dotfill\strut
  \end{tabular}%
}

\addto\captionenglish{\renewcommand*{\bibname}{Literaturliste}}

\lstset{
 inputencoding=utf8,
extendedchars=true
}

\begin{document}

\thispagestyle{empty}
\begin{center}
\vspace{3cm}
    \vspace{3cm}   {\Large \bf The Long Tale of Implementing CIC over CoC\\}\par
    \vspace{0.3cm} {\Large Technical Article\par}
    \vspace{0.3cm} {\Large Paul Lyutko, Groupoid Infinity\par}
    \vspace{4cm}   {\Large Kyiv 2016}
\end{center}

\newpage
\vspace{2cm}
\tableofcontents

\newpage
\section{Introduction: The Why}

\section{Steps: The How}
\subsection{Contexts: Curried Records}
\subsection{Setoids, Mappings}
\subsection{Categories, Functors}
\subsection{Initial and Terminal Objects}
\subsection{Inductive types and its eliminators}
\subsection{Limits and Colimits}
\subsection{Adjoint Functor Theorem}
\subsection{Categories of Dialgebras}
\subsection{Limits of Setoids}
\subsection{Creating Limits of Dialgebras}
\subsection{Simple Ornaments and Polymonial Functors}
\subsection{Getting Induction from Recursion}

\section{Examples: The Pragmatics}
\subsection{Basic Algebraic dataTypes}
\subsection{The List dataType}

\section{Advanced: There and Back Again}
\subsection{Free monad}
\subsection{Free algebraic structures}
\subsection{Existential types}
\subsection{Colimits}
\subsection{Coinductive types}
\subsection{Dependent inductive types}
\subsection{The Synthetic Universe}

\section{HoTT: The beyond}
\subsection{Infinity-Groupoids}
\subsection{Truncation}
\subsection{Higher Inductive types}
\subsection{Univalence}

\newpage
\section{References}
\begin{thebibliography}{9}

\bibitem{henk0}      Henk Barendregt \textit{The Lambda Calculus. Its syntax and semantics} 1981
\bibitem{henk1}      Henk Barendregt \textit{Lambda Calculus With Types} 2010
\bibitem{henk}       Erik Meijer, Symon Peyton Jones \textit{Henk: a typed intermediate language} 1984
\bibitem{lof}        Per Martin-Löf \textit{Intuitionistic Type Theory} 1984

\end{thebibliography}
\end{document}
